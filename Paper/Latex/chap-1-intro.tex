\chapter{Introduction}

\label{cha:intro}
 With the arrival of ChatGPT \cite{ChatGPT35Turbo}, the interest in Artificial Intelligence and Machine Learning has never been higher. 
 It has found its way into almost every field, from journalism, %\cite{Pavlik2023Collaborating}%
 research 
 % \cite{Macdonald2023Can}
 to marketing 
 % \cite{Mattas2023ChatGPT:}
 , education 
 % \cite{AlAfnan2023ChatGPT}
 , and beyond. 
 Amidst this widespread adoption across various domains, this thesis focuses on a particular niche. It delves into the creative possibilities of AI, particularly in creative text generation. Most large language models like ChatGPT, are already very capable of generating creative text, like poems, song lyrics, parodies, ... 
 Although it outputs qualitative and coherent text, it does not always comply with the rules associated with the generated poem or parody. 
 This can be seen in fig. \ref{fig:haiku-GPT} below, where ChatGPT was asked to write a haiku about how crazy pineapple on a pizza is. The result unfortunately is not a correct haiku, as it did not adhere to the 5-7-5 syllable rule. 
 The generated poem comes close but has a 5-6-5\footnote{The number of syllables of the poem was counted by using \href{https://syllablecounter.net/count}{syllablecounter.net}} syllable structure instead\footnote{When tested on other chatbots like Bard \cite{googlebard2023}, similar results are achieved which can be found in appendix \ref{app:A}}. 
 When the same prompt is given again to the chatbot (see appendix \ref{app:A}), a new haiku is generated that does adhere to the syllable rule. 
 This shows that the model does have a notion about the rules associated with haiku and tries to meet them but does not always succeed. Therefore this thesis looks into enforcing these creative constraints or rules during generation to ensure they are (almost) always fulfilled. 
 The focus will not be on generating poems, but on generating musical parodies with very similar constraints as poems but are more strictly defined.
 \begin{figure}
 \centering
     \textit{Golden waves of heat}\\
    \textit{Pineapple dances wild}\\
    \textit{Pizza's sweet embraced}\\ 
    \caption{Haiku about how crazy pineapple on a pizza is}{Generated by ChatGPT 3.5 Turbo\cite{ChatGPT35Turbo} on 12/12/2023}
    \label{fig:haiku-GPT}
 \end{figure}

 
\section{Context}
Research on musical parody generators employing language models exists, though it remains relatively limited in scope.
One of the most notable works in this field is "Weird AI Yankovic: Generating Parody Lyrics", by M. Riedl \cite{riedl_weird_2020}, which can generate parodies based on the original lyrics and a prompt to indicate the subject of the parody. 
It combines two complementing pre-existing language models and extra added logic to create a parody with the same syllable counts and rhyming scheme as the original song. 
However, this approach, while innovative, highlights the struggle to create coherent and contextually relevant text that resembles a qualitative lyric. Many existing systems in the literature (see chap. \ref{cha:3}) try to maintain the structure of the original song but often fail to preserve the lyrical essence and recognizability of the original pieces.


This is most probably caused by the language models used, which are not as performant as current models are
\footnote{Which can be seen on the \href{https://huggingface.co/spaces/HuggingFaceH4/open_llm_leaderboard}{Open LLM Leaderboard} from Huggingface, where GPT2 (The best model of the two according to M. Reidl \cite{riedl_weird_2020}) scores only an average of 28.8, whilst the top models at the time of writing score more than double with a score of 79}.
Furthermore, the method is tailor-made around both the models and the rules in their specific interpretation which makes it difficult to adapt the algorithm to use other models or to add other rules. 
All this shows that the field of musical parody generation still presents numerous opportunities for improvement and refinement in its methodologies and applications.

\textcolor{red}{[TODO]: Write a small section about the progress in constraining autoregressive models}



\section{Research Question}

\section{Approach}

\section{Overview or structure}
\textcolor{red}{[TODO]}
\section{Acknowledgements}
\subsubsection{Use of GAI}
The thesis was written with the help of GAI. Grammerly \cite{Grammarly} was used to correct any spelling mistakes and rewrite small parts of a sentence. ChatGPT (both ChatGPT 4 \cite{ChatGPT4} and ChatGPT 3.5 \cite{ChatGPT35Turbo}) was utilized to rewrite parts of paragraphs. The author confirmed no new information was added, that was not originally given. 
\subsubsection{Resources}
The computational resources and services used in this work were provided by the VSC (Flemish Supercomputer Center), funded by the Research Foundation - Flanders (FWO) and the Flemish Government - department EWI.


%%% Local Variables: 
%%% mode: latex
%%% TeX-master: "thesis"
%%% End: 
