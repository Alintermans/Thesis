\chapter{Examples of Chatbots Writing Haiku}
\label{app:A}
The following are examples of haiku written by various chatbots. They were all given the same simple prompt: \textit{"Write a haiku about how crazy pineapple on a pizza is"}. These examples were generated to demonstrate that chatbots and thus language models do not naturally follow the rules associated with certain text formats, such as poems. Specifically, the following examples are about writing haiku, which are poems consisting of three lines with a 5-7-5 syllable structure. The chatbots used in these examples have a nondeterministic character, which can be seen by giving the chatbot the same prompt twice, which will result in different results. All the syllable counts mentioned were obtained with \href{https://syllablecounter.net/count}{https://syllablecounter.net/count}. Some of them have to correct structure or are very close to it. Some of the haiku have the correct structure or are very close to it. This shows that language models already know what a haiku contains but do not always adhere to the associated constraints.

\section{ChatGPT 3.5 Turbo}
The two examples below are obtained from the ChatGPT 3.5 Turbo model\cite{ChatGPT35Turbo} from OpenAI. The first one is also presented in the introduction of this thesis to show its 5-6-5 syllable structure. When the chatbot was asked a second time, it generated a different haiku that this time adhered to the 5-7-5 syllable structure. This shows its nondeterminism and that it does have some notion about the 5-7-5 rule, but does not always know how to exactly apply it. The following examples were both generated on 12/12/2023.

\subsubsection{Example 1}
\textit{Golden waves of heat\\
Pineapple dances wild\\
Pizza's sweet embraced\\ }

\subsubsection{Example 2}
\textit{Golden sun on dough,\\
Crazy dance of pineapple \\
Pizza dreams take flight\\}

\section{Bard}
The same prompt was also tested on Bard\cite{googlebard2023} which is supported by the Gemini model \cite{gemini_team_gemini_2023}. It generated the poem below. Compared to ChatGPT the example deviates much more from the 5-7-5 syllable structure with a 6-9-6 one. It shows that the model has some notion about what a haiku is. It gives a three-line poem with a similar structure, but it has difficulty with the exact syllable count. The example was generated on 12/12/2023.

\subsubsection{Example}
\textit{Sweet, tangy fruit's delight\\ 
On savory dough, a clash so grand \\
A culinary crime\\}

\section{Llama 2}
The 70B chat model of Llama 2 \cite{touvron_llama_2023} was also tested for its haiku abilities\footnote{It was tested on the site \href{https://www.llama2.ai/}{Llama2.ai}, which hosts the vanilla versions of the LLama 2 chat models}. The model generated a haiku which is almost in line with the 5-7-5 syllable structure; it has a 5-7-4 structure according to the used syllable counter. When counting \textit{'doesn't'} as two syllables it can be seen as a correct haiku. This example was generated on 25/01/2024.

\subsubsection{Example}
\textit{Crazy pineapple\\ 
On a pizza, oh so wrong \\
Fruit doesn't belong\\}
